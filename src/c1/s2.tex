% % @author Shuning Zhang
% % @date 2020.2.17
% \documentclass[a4paper, 11pt]{book}
% \usepackage{amsfonts, amsmath, amssymb, amsthm}
% \usepackage{color}
% \usepackage{ctex}
% \usepackage{enumerate}
% \usepackage[bottom=2cm, left=2.5cm, right=2.5cm, top=2cm]{geometry}
% \usepackage{multicol}
% \newcommand{\re}{\mathrm{Re}\,}
% \newcommand{\im}{\mathrm{Im}\,}
% \begin{document}
\begin{enumerate}[(1)]
    \item % 1
        略.
    \item % 2
        \begin{enumerate}
            \item % a
                \begin{proof}
                    $\re(iz) = \re(-y+ix) = -y = -\im(x+iy) = -\im{z}$.
                \end{proof}
            \item % b
                \begin{proof}
                    $\im(iz) = \im(-y+ix) = x = \re(x+iy) = \re{z}$.
                \end{proof}
        \end{enumerate}
    \item % 3
        \begin{proof}
            \begin{align*}
                (z+1)^2 &= (z+1)(z+1) \\
                &= (x+iy+1)(x+iy+1) \\
                &= (x+1)^2 + 2(x+1)yi - y \\
                &= x^2+2x+1 + 2xyi+2yi - y \\
                &= x^2+2xyi-y + 2(x+iy) + 1 \\
                &= z^2 + 2z + 1. \qedhere
            \end{align*}
        \end{proof}
    \item % 4
        略.
    \item % 5
        略.
    \item % 6
        略.
    \item % 7
        略.
    \item % 8
        \begin{proof}
            \begin{gather*}
                -(iy) = -((0,1)(y,0)) = -(0,y) = (0,-y), \\
                (-i)y = (0,-1)(y,0) = (0,-y), \\
                i(-y) = (0,1)(-y,0) = (0,-y). \qedhere
            \end{gather*}
        \end{proof}
    \item % 9
        略.
    \item % 10
        \begin{align*}
            z^2+z+1=0 &\Rightarrow (x,y)(x,y) + (x,y) + (1,0) = (0,0) \\
            &\Rightarrow (x^2-y^2,2xy) + (x+1,y) = (0,0) \\
            &\Rightarrow (x^2-y^2+x+1,2xy+y) = (0,0) \\
            &\Rightarrow
            \begin{cases}
                x^2-y^2+x+1 = 0, \\
                y(2x+1) = 0.
            \end{cases}
        \end{align*}
        从第二个方程解得 $x=-\dfrac12$, $y=0$. 当 $y=0$ 时, 第一个方程则变为
        \[
            x^2+x+1 = 0,    
        \]
        无解. 因此 $y=0$ 不是方程的解. 将 $x=-\dfrac12$ 带回第一个方程, 得到
        \[
            y^2 = \frac{3}{4}.    
        \]
        解得 $y = \pm\dfrac{\sqrt{3}}{2}$. 因此 $z=\displaystyle{\left(-\frac12,\pm\frac{\sqrt{3}}{2}\right)}$.
\end{enumerate}
% \end{document}
