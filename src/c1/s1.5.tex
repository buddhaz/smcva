% @author Shuning Zhang
% @date 2020.2.17
\documentclass[a4paper, 11pt]{book}
\usepackage{amsfonts, amsmath, amssymb, amsthm}
\usepackage{color}
\usepackage{ctex}
\usepackage{enumerate}
\usepackage[bottom=2cm, left=2.5cm, right=2.5cm, top=2cm]{geometry}
\usepackage{multicol}
\newcommand{\re}{\mathrm{Re}\,}
\newcommand{\im}{\mathrm{Im}\,}
\begin{document}
\begin{enumerate}
\item % 1
    \begin{enumerate}
        \item % a
            \begin{proof}
                \begin{align*}
                    \overline{\overline{z}+3i} &= \overline{x+(3-y)i} \\
                    &= \overline{x-(3-y)i} \\
                    &= x+iy - 3i \\
                    &= z - 3i. \qedhere
                \end{align*}
            \end{proof}
        \item % b
            \begin{proof}
                \begin{align*}
                    \overline{iz} &= \overline{i(x+iy)} \\
                    &= \overline{-y+ix} \\
                    &= -y-ix \\
                    &=(-i)(x-iy) \\
                    &=-i\overline{z}. \qedhere
                \end{align*}
            \end{proof}
        \item % c
            \begin{proof}
                $\overline{(2+i)^2} = \overline{i^2+4i+4} = \overline{3+4i} = 3-4i$.
            \end{proof}
        \item % d
    \end{enumerate}
\item % 2
    \begin{enumerate}
        \item % a
            $
                \re(\overline{z} - i) = \re(x-yi-i)
                = \re(x - (y+1)i)
                = x.
            $
        \item % b
            因为 $|2z-i| = |2x + 2yi - i| = |2x + (2y-1)i| = \sqrt{(2x)^2 + (2y-1)^2} = 4$,
            故
            \[
                (2x)^2 + (2y-1)^2 = 16
                \Rightarrow
                x^2 + \left(y-\frac12\right)^2 = 4.  
            \]
    \end{enumerate}
\item % 3
    略.
\item % 4
    \begin{enumerate}
        \item $\overline{z_1z_2z_3} = \overline{z_1z_2}\,\overline{z_3} = \overline{z_1}\,\overline{z_2}\,\overline{z_3}$;
        \item $\overline{z^4} = \overline{zzzz} = \overline{z}\,\overline{z}\,\overline{z}\,\overline{z} = \overline{z}^4$.
    \end{enumerate}
\item % 5
    因为
    \[
        \left|\frac{z_1}{z_2}\right|^2 = \left(\frac{z_1}{z_2}\right)\overline{\left(\frac{z_1}{z_2}\right)} = \left(\frac{z_1}{z_2}\right)\left(\frac{\overline{z_1}}{\overline{z_2}}\right) = \frac{z_1\overline{z_1}}{z_2\overline{z_2}} = \frac{|z_1|^2}{|z_2|^2},    
    \]
    故 $\displaystyle{\left|\frac{z_1}{z_2}\right| = \frac{|z_1|}{|z_2|}}$.
\item % 6
    \begin{enumerate}
        \item \begin{proof}
            $\displaystyle{
                \overline{\left(\frac{z_1}{z_2z_3}\right)} = \frac{\overline{z_1}}{\overline{z_2z_3}} = \frac{\overline{z_1}}{\overline{z_2}\,\overline{z_3}}
            }$;
        \end{proof}
        \item \begin{proof}
            $\displaystyle{
                \left|\frac{z_1}{z_2z_3}\right| = \frac{|z_1|}{|z_2z_3|} = \frac{|z_1|}{|z_2||z_3|}
            }$.
        \end{proof}
    \end{enumerate}
\item % 7
    \begin{proof}
        $\displaystyle{
            \left| \frac{z_1+z_2}{z_3+z_4} \right| = \frac{|z_1+z_2|}{|z_3+z_4|} \leqslant \frac{|z_1|+|z_2|}{|z_3+z_4|} \leqslant \frac{|z_1|+|z_2|}{|z_3|-|z_4|}
        }$.
    \end{proof}
\item % 8
\item % 9
    略.
\item % 10
    \begin{proof}
        \begin{align*}
            \because |z^4 - 4z^2 + 3| &= |(z^2 - 1)(z^2 - 3)| \\
            &= |z^2 - 1||z^2 - 3| \\
            &\geqslant (|z^2| - 1)(|z^2| - 3) \\
            &= (|z|^2 - 1)(|z|^2 - 3) \\
            &= 3 \cdot 1 = 3, \\
            \therefore \frac{1}{|z^4 - 4z^2 + 3|} &\leqslant \frac13. \qedhere
        \end{align*}
    \end{proof}
\item % 11
    \begin{enumerate}
        \item \begin{proof}
            \begin{align*}
                \text{$z = x + iy$ 是一个实数} &\Leftrightarrow y = 0 \\
                &\Leftrightarrow \overline{z} = (x, 0) \\
                &\Leftrightarrow z = \overline{z}. \qedhere
            \end{align*}
        \end{proof}
        \item \begin{proof}
            若 $z$ 是实数, 由 (a) 可知 $z = \overline{z} \Leftrightarrow z^2 = \overline{z}^2$.
            若 $z$ 是纯虚数, 则有
            \[
                z^2 = (iy)^2 = i^2y^2 = (-i)^2y^2 = (-iy)^2 = \overline{z}^2. \qedhere 
            \]
        \end{proof}
    \end{enumerate}
\item % 12
    略.
\item % 13
\item % 14
\item % 15
\item % 16
\end{enumerate}
\end{document}
